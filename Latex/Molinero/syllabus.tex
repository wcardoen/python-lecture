\documentclass[11pt]{article}
\usepackage{aeguill}                            % Fonts in acrobat reader
\usepackage[german,dutch,french,english]{babel} % Language to be used
\usepackage{color}
\usepackage{hyperref} 
\usepackage[rgb,dvipsnames]{xcolor} 
\definecolor{OliveGreen}{rgb}{0,0.6,0}
\hypersetup{
  pdftitle={Introduction to Python},
  pdfauthor={Wim R. Cardoen, PhD},
  pdfpagelayout=SinglePage,
  bookmarks=false,
  colorlinks=true,
  urlcolor=blue,    
  linkcolor=red,
  citecolor=OliveGreen
}

\begin{document}
\thispagestyle{empty}
\title{Introduction to Python}
\author{Wim R. Cardoen, PhD \\
        University of Utah\\
        Center for High Performance Computing \\ 
        155 S. 1452 E. Rm. 420\\ 
        Salt Lake City, UT 84112-0190 \\
        USA\\
        \texttt{wim.cardoen@utah.edu}}
%\renewcommand{\today}{Sep 10, 2014} 
\renewcommand{\labelitemii}{$\star$}
\maketitle\thispagestyle{empty}
\pagebreak
% ----------------------------------------------------------------
\pagestyle{plain}
\pagenumbering{arabic}
\setcounter{page}{1}
\section*{Introduction}
In the coming three lectures I will give an introduction to the use of Python in scientific computing. 
During the first two lectures I will cover the core of the Python language.
In the last lecture the basics of numpy, scipy and matplotlib (if time permits) will be addressed.

\renewcommand \thesection{\Roman{section}}
\section{Python distribution}
During the lectures I will be using the Anaconda Scientific Python distribution.
The distribution can be downloaded from \href{https://store.continuum.io/cshop/anaconda/}{here}.

If you plan to attend the lectures you \textbf{must} install the Anaconda distribution 
\textbf{before} the start of the first lecture. Due to time constraints we do not have time to 
do the installation in class.
If you face a problem with the installation, please send an email to the 
\href{mailto:issues@chpc.utah.edu}{CHPC Helpdesk} (Subject of the email: Anaconda install).

\section{When/where?}
All Python lectures will take place in the INSCC building from 1.00 pm to 3.00 pm
\begin{itemize}
 \item Mon. 09/15/2014  Rm. $407$ (a.k.a. INSCC Training Lab)
 \item Mon. 09/22/2014  Rm. $345$
 \item Wed. 10/01/2014  Rm. $105$ (a.k.a. INSCC Auditorium)
\end{itemize}


\section{Topics to be covered}
\begin{itemize} 
 \item Introduction to IPython
 \item Why Python?
 \item Variables, assignments, operators
 \item Control structures
 \item Functions
 \item Modules
 \item Standard library 
 \item Exceptions
 \item Object-Oriented programming in Python
 \item $\ldots$
 \item Numpy, Scipy \& a taste of Matplotlib (if time permits)
\end{itemize}

\section{Useful links}
\begin{itemize} 
 \item Online Tutorials:
    \begin{itemize} 
       \item \url{https://docs.python.org/2/} 
       \item \url{http://www.numpy.org/}
       \item \url{http://www.scipy.org/}
    \end{itemize} 
 \item Books
    \begin{itemize} 
       \item Python Essential Reference\,\cite{BEAZLEY:2009a}
       \item $\ldots$
    \end{itemize} 
 \item IDEs
    \begin{itemize}
       \item \href{http://code.google.com/p/spyderlib/}{Spyder}
       \item \href{http://www.eclipse.org/}{Eclipse} (You need to install the Python plugin)
       \item \href{http://www.jetbrains.com/pycharm/}{PyCharm} (The Community Edition is free) 
    \end{itemize}
    The Spyder IDE is installed within the Anaconda distribution and will be introduced during the class.
\end{itemize}

% ------------------------------ Bibliography -----------------------
\bibliographystyle{plain}
\bibliography{syllabus}
\end{document}

