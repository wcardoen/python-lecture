\documentclass[11pt]{article}
\usepackage{aeguill}                            % Fonts in acrobat reader
\usepackage[german,dutch,french,english]{babel} % Language to be used
\usepackage{color}
\usepackage{hyperref} 
\usepackage[rgb,dvipsnames]{xcolor} 
\definecolor{OliveGreen}{rgb}{0,0.6,0}
\hypersetup{
  pdftitle={Introduction to Python - projects},
  pdfauthor={Wim R. Cardoen, PhD},
  pdfpagelayout=SinglePage,
  bookmarks=false,
  colorlinks=true,
  urlcolor=blue,    
  linkcolor=red,
  citecolor=OliveGreen
}

\begin{document}
\thispagestyle{empty}
\renewcommand{\labelitemii}{$\star$}
\pagebreak
% ----------------------------------------------------------------
\pagestyle{plain}
\pagenumbering{arabic}
\setcounter{page}{1}
\section*{Exercises}
At the end of this course you should be able to tackle a lot of computational problems.
You should be able to write your own Molecular Dynamics (MD) or Monte Carlo (MC) code 
to perform molecular simulations. Just do it!.

The
The following problems are interesting programming exercises where you can apply a lot of 
the techniques that you have been learning during the Python lectures.

\begin{itemize}
\item Ewald summation \newline
The Ewald\,\cite{EWALD:1921a,DELEEUW:1980a,DELEEUW:1980b,DELEEUW:1983a} summation 
has been developed to calculate the Coulomb interaction for systems with periodic boundary conditions. 
Write your own Python code to calculate the energy, forces 
(using analytical derivatives - you can check your code using numerical derivates) and 
virial for a 3D periodic system. Calculate the Madelung constant for a NaCl lattice using your Ewald code. 
Observe the "convergence" of the Ewald summation.  

\textbf{Extra challenge}:\newline
In the latter nineties the Smooth Particle Mesh Ewald\,\cite{ESSMANN:1995a} (SPME) 
method was developed as an alternative way to calculate the Ewald sums. 
Instead of doing a regular sum the charges are first mapped (using a spline interpolation) 
into a regular equidistant grid. The summation can then be performed with a Fast Fourier 
Transform (FFT).

\item Ground state of the Heisenberg spin lattice\newline
The Heisenberg Hamiltonian is used to model ferromagnetic and antiferromagnetic systems at low temperature.
The atoms (each having spin $S_i$ are placed on the vertices of the lattice. 
The interaction is given by the (Heisenberg) Hamiltonian:
\begin{eqnarray}
       \hat{H} & = & \frac{J}{2} \, \sum_{i} \hat{S}_{i}  \hat{S}_{i+1}
\end{eqnarray}
Write your own Python code to calculate the ground state wave function and its energy for a 1D and 2D lattices
where the spin of each atom is: $\frac{1}{2}$, $1$, $2$. 

FYI: Hans Bethe\,\cite{BETHE:1931a} worked out an analytical solution (using the Bethe Ansatz)
for the infinite 1D lattice for spin $\frac{1}{2}$ particles.

\textbf{Extra challenge}:\newline
The Hilbert space for this problem scales exponentially with the size of the system (in casu $(2\,S+1)^N$ where $N$ 
stands for the number of vertices on the lattice and $S$ stands for the atom's spin). 

If you want to find the ground state energy and wave function of large spin chains, 
you need to implement iterative eigenvalue methods for large sparse matrices (using 
e.g.~the Lanczos\,\cite{CULLUM:2002a} or Davidson\,\cite{DAVIDSON:1975a} algorithms).

\item Finding the transition state on a PES\newline
The transition state is an important feature in the study of chemical reactions. 
Cerjan and Miller\,\cite{CERJAN:1979a} wrote one of the earliest papers on how to find a first-order transition state 
using the eigenmode following technique.
Write your own Python code and find the minima/transition states of the M\"uller-Brown surface\,\cite{MUELLER:1979a}, which 
has the following functional form:
\begin{eqnarray}
   f(x,y) & = & \sum_{i=1}^{4} A_i\, \exp \Large[ a_i (x-x_0^i)^2 \,+\, b_i (x-x_0^i)(y-y_0^i)\, + \\
          &   &                                   c_i (y-y_0^i)^2 \Large ]                                   
\end{eqnarray}
\item Suggestions?

\end{itemize}

% ------------------------------ Bibliography -----------------------
% Check also: /uufs/chpc.utah.edu/common/home/u0253283/public_html/private/phys/statmech/md/ewald
\bibliographystyle{h-physrev}
\bibliography{string,syllabus}
\end{document}

